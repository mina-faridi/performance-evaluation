\documentclass{article}

\usepackage{amsthm,amsmath,amssymb}

\begin{document}
	
	
\section*{2.1}
\begin{align}
p(X=0) & = 0.18 \nonumber \\
p(X=1) & = 0.27 \nonumber \\
p(X=2) & = 0.34 \nonumber \\
p(X=3) & = 0.14 \nonumber \\
p(X=4) & = 0.07 \nonumber \\
p(X>4) & = 0 \nonumber \\
\end{align}

\section*{2.5}
(a) X: number of the packets in first slot\\
Y: number of the packets in second slot\\
\begin{equation*}
	Y = X - min \{X,c\}
\end{equation*}
\\ for k=0,1,...,b-1
\begin{equation*}
	p_X(k) = e^{-\lambda \frac{\lambda^k}{k!}}
\end{equation*}

\begin{equation*}
	p_X(b) =  \sum_{k=b} ^{\infty} e^{-\lambda \frac{\lambda^k}{k!}} = 1 -  \sum_{k=0} ^{b-1} e^{-\lambda \frac{\lambda^k}{k!}}
\end{equation*}

\begin{equation*}
	p_{Y}(k)=p_{X}(k+c)=e^{-\lambda} \frac{\lambda^{k+c}}{(k+c) !}, \quad k=1, \ldots, b-c-1
\end{equation*}

\begin{equation*}
p_{Y}(b-c)=p_{X}(b)=1-\sum_{k=0}^{b-1} e^{-\lambda} \frac{\lambda^{k}}{k !}
\end{equation*}


(b) It is equal to the probability of more than b packets being generated by the source
\begin{equation*}
\sum_{k=b+1}^{\infty} e^{-\lambda} \frac{\lambda^{k}}{k !} = 1-\sum_{k=0}^{b} e^{-\lambda} \frac{\lambda^{k}}{k !}
\end{equation*}




\section*{2.9}
for k=1,...,n: 
\begin{equation*}
\frac{p_{X}(k)}{p_{X}(k-1)}=\frac{\left(\begin{array}{c}n \\ k\end{array}\right) p^{k}(1-p)^{n-k}}{\left(\begin{array}{c}n \\ k-1\end{array}\right) p^{k-1}(1-p)^{n-k+1}}=\frac{(n+1)p-kp}{k-kp}
\end{equation*}
for $k<=k^*$
\begin{equation*}
 k-kp<= (n+1)p-kp 
\end{equation*}
so the ratio is more than one and $p_X(k)$ is monotically nondecreasing\\
for $k>k^*$
\begin{equation*}
 k-kp> (n+1)p-kp 
\end{equation*}
so the ratio is less than one and  $p_X(k)$ is monotically decreasing



\section*{2.13}
N: number of natural\\
G: number of girls\\
$G=N+2$\\
\begin{equation*}
p_{N}(k)= \begin{cases}\left(\begin{array}{l}
5 \\
k
\end{array}\right) \cdot\left(0.5\right)^{5} & \text  0 \leq k \leq 5 \\
0 & \text { else }\end{cases}
\end{equation*}
$$
p_{G}(g)=\sum_{n: n+2=g} p_{N}(n)=p_{N}(g-2)
$$
$$
p_{G}(g)= \begin{cases}\left(\begin{array}{c}
5 \\
g-2
\end{array}\right) \cdot\left(0.5\right)^{5}, & \text  2 \leq g \leq 7 \\
0 & \text { else }\end{cases}
$$

\section*{2.17}
$$Y=32+\frac{9 X}{ 5}$$
$$
E[Y]=32+\frac{9 E[X]}{ 5}=32+18=50
$$
$$
var(Y)=(\frac{9}{ 5})^{2} var(X)
$$
$$
var(X)=100
$$
so
$$
var(Y)=18
$$
$$
range=[50-18,50+18]=[32,68]
$$
\section*{2.21}
$$
E[X]=\sum_{k=1}^{\infty} 2^{k} \cdot 2^{-k}=\sum_{k=1}^{\infty} 1=\infty
$$
the game has gain in any case\\
so it doesn't matter how much we pay
\section*{2.25}
(a)
$$
p_{I, J}(i, j)= \begin{cases}\frac{1}{\sum_{k=1}^{n} m_{k}}, & \text { if } j \leq m_{i} \\ 0, & \text { otherwise. }\end{cases}
$$
$$
p_{I}(i)=\sum_{j=1}^{m} p_{I, J}(i, j)=\frac{m_{i}}{\sum_{k=1}^{n} m_{k}} \quad i=1, \ldots, n
$$
$$
p_{J}(j)=\sum_{i=1}^{n} p_{I, J}(i, j)=\frac{l_{j}}{\sum_{k=1}^{n} m_{k}} \quad j=1, \ldots, m
$$
(b)
$$
\sum_{j=1}^{m_{i}}\left(p_{i j} a+\left(1-p_{i j}\right) b\right)
$$

\section*{2.41}
(a)\\
$p=0.02$\\
$n=250$\\
$E[X]=n p=250 \cdot 0.02=5$\\
$$
P(X=5)=\left(\begin{array}{c}
250 \\
5
\end{array}\right)(0.02)^{5}(0.98)^{245}=0.1773
$$
(b)
$\lambda=n p=5$\\
$$
e^{-\lambda} \frac{\lambda^{5}}{5 !}=0.1755
$$

(c)
M: money paid in a year

$$
E[M]=\sum_{i=1}^{5} 50 E\left[M_{i}\right]
$$
$$
P\left(M_{i}=m\right)= \begin{cases}0.98 & \text m=0 \\ 0.01 & \text m=10 \\ 0.006 & \text m=20 \\ 0.004 & \text  m=50\end{cases}
$$
$$
E\left[m_{i}\right]=0.01 \times 10+0.006 \times 20+0.004 \times 50=0.42
$$
$$
var\left(M_{i}\right)=E\left[M_{i}^{2}\right]-\left(E\left[M_{i}\right]\right)^{2}=13.22
$$
$$
E[M]=250E\left[M_{i}\right]=105,
$$
$$
var(M)=250 var\left(M_{i}\right)=3305 .
$$
(d)
$$
var = \frac{p(1-p)}{250}
$$
p should be between [0,1] and satisfy:
$$
(p-0.02)^{2} \leq \frac{25 p(1-p)}{250}
$$
\end{document}